\documentclass[english,12pt,letter]{article}

\usepackage{amsthm}
\usepackage[latin9]{inputenc}
\usepackage{babel}
\usepackage[hmargin=0.9in,vmargin=1.25in]{geometry}
\usepackage{graphicx}
\usepackage{subfigure}
\usepackage[colorlinks=true,citecolor=blue,urlcolor=blue]{hyperref}
\usepackage{amsmath}
\usepackage{amssymb,amsfonts}

\newtheorem{thm}{Theorem}
\newtheorem{lem}{Lemma}
\newtheorem{cor}{Corollary}
\newtheorem{dfn}{Definition}

\newcommand{\Rnot}{\sigma_0}
\newcommand{\Sinf}{x_\infty}
\newcommand{\dom}{{\mathcal D}}
\newcommand{\R}{{\mathbb R}}
\newcommand{\xopt}{x_\text{opt}}
\newcommand{\yopt}{y_\text{opt}}
\newcommand{\ymax}{y_\text{max}}
\newcommand{\ifr}{F}

\DeclareMathOperator\supp{supp}

\begin{document}
\title{A model for forecasting the COVID-19 pandemic and assessing responses to it}
\author{
  David I. Ketcheson\thanks{Computer, Electrical, and Mathematical Sciences \& Engineering Division,
King Abdullah University of Science and Technology, 4700 KAUST, Thuwal
23955, Saudi Arabia. (david.ketcheson@kaust.edu.sa)}
}
\maketitle

\abstract{We describe a model for predicting the death toll over time
due to the ongoing COVID-19 epidemic.  The model is based on recorded
numbers of deaths and makes predictions of daily numbers of new infected
persons and deaths.  The effect of non-pharmaceutical interventions (NPIs) is
modelled empirically, based on recent trends in the death rate.  The model can
also be used to study counterfactual scenarios based on hypothetical NPI
scenarios.
}

\section{Overview of the model}

The central part of the forecasting model is the SIR model of Kermack \&
Mckendrick \cite{kermack1927contribution}. The SIR model is a compartmental model
that divides the population into three groups:
\begin{itemize}
  \item S(t): Susceptible (those who have not yet been infected)
  \item I(t): Infected (those who can currently spread the disease)
  \item R(t): Recovered (those who are now immune and cannot contract or spread the disease)
\end{itemize}
The model takes the form of a system of three ordinary differential
equations:
\begin{align}
\frac{dS}{dt} & = -\beta I \frac{S}{N} \\
\frac{dI}{dt} & = \beta I \frac{S}{N}-\gamma I \\
\frac{dR}{dt} & = \gamma I
\end{align}
Here N is the total population.  Since R = N - S - I, the system can be modelled with just
the first two equations.

We extend this model by incorporating a time-dependent intervention parameter $q(t)$,
which models the change in the contact rate $\beta$ due to social distancing, school
and work closures, and other such measures.  This extension takes the following
form:
\begin{align}
\frac{dS}{dt} & = -(1-q(t))\beta I \frac{S}{N} \\
\frac{dI}{dt} & = (1-q(t))\beta I \frac{S}{N}-\gamma I \\
\frac{dR}{dt} & = \gamma I
\end{align}
A value $q(t)=0$ corresponds to no intervention, while $q(t)=1$ would correspond
to a complete elimination of any human contact.  This parameter can model both
population-wide measures and those specifically targeting infected or suspected
individuals.

\subsection{Details of forecast algorithm}
The model is applied to individual countries or states.  For each region, it
involves the following steps:
\begin{enumerate}
    \item Get recorded numbers of deaths per day since the start of the
            epidemic, denoted $D_i$, either from the JHU data set or the NYT data set.
    \item Infer the number of daily new infections $I_i$ for dates in the past,
            based on an assumed distribution of the time from infection to
            death.  Let $p(k)$ denote the probability that a newly infected
            person dies $k$ days after infection.  We have
            $$
                \sum_k p(k) = \ifr
            $$
            where $\ifr$ is the infection fatality ratio, currently estimated
            to be about 0.006 for COVID-19.
            The quantities $D_i$ and $I_i$ are related by
            $$
                 D_i = \sum_{k=0}^\infty I_{i-k} p(k).
            $$ 
            The problem of solving this equation for $I_i$ given $D_i$ is 
            known as deconvolution, and it is ill-conditioned.  Due to non-smoothness
            of real-world recorded death rates, the deconvolved signal is
            typically  highly oscillatory and includes negative values.
            A good regularization for this problem is the subject of ongoing
            work.  In the meantime, the model estimates a mean time to death $m$
            based on the recent death rate trend and then sets
            $$
                I_{i-m} = D_i / \ifr.
            $$
            For a uniform (in time) distribution of deaths $D_i$ the mean time
            would be $m=17$ days, but the actual value of $m$ at a given time
            and place may be shorter (if the epidemic is growing) or longer (if the
            epidemic is diminishing).
            This inference step above gives values of $I_i$ up to $m$ days in the past. 
            The cumulative number of recovered individuals at day $i$ is then computed
            as
            $$
                R(i) = \sum_{j=0}^n I_j (1-e^{-\gamma(n-j)}).
            $$
            The number of actively infected persons on day $n$ is then
            $$
                I(i) = (\sum_0^n I_i) - R^C_i.
            $$
            The number of susceptibles on day $i$ is
            $$
                S(i) = N - R(i) - I(i).
            $$
    \item   Let $i=0$ denote the present day.  The SIR model is run starting from day $i=-m$
            up to $i=0$.  The initial data is given by the inferred values above.
            The value of $q(t)$ is determined by fitting the model output to the recorded numbers
            of deaths.  This fitting is done in two ways.  In the first way, a constant value is assumed
            $q(t)=q_0$.  In the second way, a linear function is assumed: $q(t) = q_1 + \delta t$.
            In the second fit, there is a penalty for using a non-zero slope; in both fits,
            $q(t)$ is constrained to lie in $[0,1]$.  Note that the second fit gives a value
            of $q_1$ for the intervention at the present day.  The value $q_1$ will be used as the
            assumed intervention effectiveness in the next step, while the two fits are used
            to give a range of plausible intervention parameters 
            \begin{align} \label{qrange}
                [\min(q_0,q_1)-0.2, \max(q_0,q_1)+0.2],
            \end{align}
            used to generate a range of model outcomes.  This range is also constrained to lie in $[0,1]$.
    \item   Finally, the output values of the last step at day $i=0$ are used as initial data for the forecast.
            The primary forecast is generated using the intervention effectiveness value $q_1$, while
            the range is generated using the set of values \eqref{qrange}.
\end{enumerate}
Forecasts can also be generated using some assumed future intervention policy that changes in time,
but at present we are not generating any such forecasts.

\subsection{Details of counterfactual no-intervention scenario algorithm}
The model can also be used to generate a hypothetical scenario starting at some time in the past.
The main purpose of this is to examine what might have happened in a given region in the absence
of intervention.  This involves the following steps:
\begin{enumerate}
    \item Determine a starting date for the model.  This should be a date before substantial
            intervention started, but after a statistically significant number of deaths
            had occurred.  For some regions (those that are very small or where the epidemic
            started very late) these two requirements may be contradictory, and no reasonable
            modeling can be done.  At present, the model finds a the earliest day on which
            the cumulative death toll was greater than 50.
    \item Determine initial numbers of susceptible, infected, and recovered for the starting
            date.  The number of infected is assumed to be about $\alpha D_i/\ifr$ where
            $D_i/ifr$ is the number infected $m$ days before the start date and
            $\alpha\approx 10$ accounts for growth of the epidemic during those $m$ days.
            There is an extremely high level of inherent uncertainty in estimating this initial data,
            and the model outcome is highly sensitive to it.
    \item This initial data is fed into the SIR model without intervention.
\end{enumerate}

\section{Data sources}
Daily deaths:
\begin{itemize}
    \item For countries: \url{https://github.com/CSSEGISandData}
    \item For states: \url{https://github.com/nytimes}
\end{itemize}

Population data: \url{https://population.un.org/wpp/Download/Standard/Population/}

\section{Key parameter values}
Here we discuss the key parameters and give the values used in the model.
The discussion here is relatively brief; additional references used in
determining parameter values for the model are listed for now at
\url{https://github.com/ketch/covid-blog-posts/wiki/Parameter-estimates}.

{\bf The basic reproduction number $\Rnot$.}  This is given by $\beta/\gamma$ and
is the average number of new infections that would be generated by one
infected individual in a fully-susceptible population (in the absence of
intervention).  Early estimates of the basic reproduction number were
low, with some even less than two \cite{wu2020estimating},
but more recent estimates are closer to four \cite{siwiak2020single}.
A survey of values for the basic reproduction number can be
found in \cite{liu2020reproductive}.  We take $\beta=0.27$ and
$\gamma=0.07$, giving $\Rnot\approx3.85$, similar to the value used in the
March 30 Imperial College report \cite{flaxman2020report}.

{\bf The doubling time.}  This is the time it would take for the number of
infected persons to double, and is given in the SIR model by $\log(2)/(\beta-\gamma)$;
with our parameters this is about $3.47$ days, which is consistent with
observed patterns in many countries prior to intervention.

{\bf The infection fatality ratio $\ifr$.}  This is the fraction of
infected persons that die from COVID-19.  There is no way to determine
this directly because we don't know the total number of infected
persons; we can instead calculate the case fatality ratio based on
confirmed cases, which is much higher, especially in the case of COVID-19,
due to the large number of mild and asymptomatic cases.
Furthermore, the IFR for COVID-19 is known to vary strongly with age.
We use age-specific IFR estimates from \cite{verity2020estimates};
these have been reinforced by our own (unpublished) analysis of data
from Spain and the USA.  These are
combined with UN demographic data for each country to determine an
effective IFR:
$$
    \ifr_\text{eff} = \frac{1}{N} \sum_j \ifr_j N_j
$$
where $\ifr_j$ is the IFR for age group $j$ and $N_j$ is the
population in age group $j$.
For regions in which we lack detailed demographic data, 
we use a value of 0.6\% based on estimates from 
\cite{russell2020estimating,wu2020estimating,verity2020estimates}.

\section{Sources of error and uncertainty}
The adage that "all models are wrong, but some models are useful"
is certainly applicable here.  The difficulty of modeling an epidemic
of a new disease while it is ongoing is high, and all results should
be taken as rough estimates.  Here we list some of the primary sources
of error and uncertainty.

\begin{itemize}
    \item Uncertainty in the key parameters $\beta, \gamma, \ifr$.
            Published scientific estimates of each of these values vary by
            at least a factor of 2, and two of these parameters enter
            into the exponent of the exponential growth of the epidemic.
            Therefore, even small uncertainties in these parameters can lead to 
            very large uncertainties in the model outputs.
    \item Inaccuracies in the input data.  Our model intentionally disregards
            confirmed case numbers because they are a very poor indicator of
            real infections.  Numbers of deaths are typically much more accurate,
            but still contain substantial errors.  It may be expected that in
            some countries these numbers are deliberately manipulated for
            political purposes; in this case {\bf it must be emphasized that
            the model outputs will be completely meaningless.}  We believe this
            to be the case for instance in Iran and Russia, where evidence suggests
            the death toll has been drastically under-reported.  But even
            in countries where the responsible parties try to report accurate
            numbers, real death rates are often significantly higher than those
            reported.  For instance, comparison of excess mortality in Spain
            with published COVID-19 death numbers suggests that the death toll
            may be underestimated by about 1/3.
    \item Inaccurate model assumptions.  The SIR model is one of the simplest
            epidemiological models.  One of its key assumptions is that
            of homogeneous mixing among the population -- i.e., that any two
            individuals are equally likely to have contact.  This is of course
            far from true in the real world.  More detailed models involving
            spatial structure and human networks can partially remedy this
            deficiency.  We have stuck to the SIR model so far because it seems
            that the parameter uncertainties already listed above -- which will
            impact any epidemiological model -- are more signicant than this
            modeling error.
    \item Modeling the impact of intervention.  Other models, such as that of
            \cite{flaxman2020report}, have attempted to explicitly model the
            effect of officially-announced intervention policies, lowering
            the contact rate by some amount starting when the policy is put
            in place.  This requires an inherently uncertain estimation of the
            quantitative impact of a given policy, which will certainly vary
            in time and between different societies.  To avoid this, we use
            an empirical assessment of intervention, choosing a contact rate
            that reproduces the data.  This contact rate is somewhat sensitive
            to the noisy daily death rate data.  Additionally, the model is
            slow to adapt to changes in intervention policy, since their effects
            do not show up in the death rate until at least 2-3 weeks after
            they are implemented.
    \item Assumptions about future policy.  Our primary forecast assumes that
            the current level of intervention will be maintained over the
            entire forecast period.  For long-range forecasts, this is very
            unlikely.  The model is not intended to show what will likely
            happen in reality, since future intervention policy is a matter
            of politics and not susceptible to detailed mathematical modeling.
            Rather, the forecast values should be interpreted as showing what
            will likely happen {\em if the current intervention stays in place.}
            We plan to release forecasts for other intervention scenarios in
            the future.
\end{itemize}


\bibliographystyle{plain}
\bibliography{refs}

\end{document}
